\section{The create command}

\subsection{How it works}
\textbf{\textcolor{blue}{create}}: takes as an input the size of the file system we want to create.\\

When calling the create command, memory will be allocated on the heap for both the directory and the file array since they are dynamic, and then a file system structure is filled with the initial data. Eventually the new file system will have: 
\begin{itemize}
\item max\_size will be size*1000*1000 (size is passed to create in MB and its multiplied by ${10}^{6}$ in order to convert it into bytes).
\item the directory and file's number is initialized to 0.
\item the current size of the file system will updated to the size of the Superblock plus the size of the directory and file arrays.
\item using add\_directory function, root directory "/" with index -2 is added to the file system after allocating memory for 1 Directory struct in the directory array.
\end{itemize}

At the end the FileSystem struct will be stored on the disk using put\_FS function and all the allocated memory will be freed up.

\newpage
\subsection{Testing}
\lstset{language=bash,caption={create command},label=code:create-command}
\begin{lstlisting}
$ vmhFS /tmp/tmpFS create 10
FileSystem created
Size: 80 B
Max size: 10000000 B
Files: 0
Directories: 1
\end{lstlisting}

\lstset{language=bash,caption={File system on the disk},label=code:file-system-on-the-disk}
\begin{lstlisting}
$ stat /tmp/tmpFS 
  File: /tmp/tmpFS
  Size: 10000000        Blocks: 8          IO Block: 4096   regular file
Device: 10302h/66306d   Inode: 5898250     Links: 1
Access: (0664/-rw-rw-r--)  Uid: ( 1000/  michel)   Gid: ( 1000/  michel)
Access: 2022-12-01 14:31:22.790783924 +0100
Modify: 2022-12-01 15:13:15.805815135 +0100
Change: 2022-12-01 15:13:15.805815135 +0100
 Birth: 2022-12-01 13:17:22.874251973 +0100
\end{lstlisting}

\newpage